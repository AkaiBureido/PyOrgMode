% Created 2010-12-14 Tue 21:35
\documentclass[11pt]{article}

\usepackage[utf8]{inputenc}
\usepackage[T1]{fontenc}
\usepackage{fixltx2e}
\usepackage{graphicx}
\usepackage{longtable}
\usepackage{float}
\usepackage{wrapfig}
\usepackage{soul}
\usepackage{t1enc}
\usepackage{textcomp}
\usepackage{marvosym}
\usepackage{wasysym}
\usepackage{latexsym}
\usepackage{amssymb}
\usepackage{hyperref}
\tolerance=1000
\providecommand{\alert}[1]{\textbf{#1}}

\title{PyOrgMode}
\author{}
\date{14 December 2010}

\begin{document}

\maketitle

\setcounter{tocdepth}{3}
\tableofcontents
\vspace*{1cm}
\section{PyOrgMode}
\label{sec-1}
\subsection{Tools}
\label{sec-1_1}

\texttt{Tangle} (Export the PyOrgMode.py file)
\subsection{Documentation}
\label{sec-1_2}
\subsubsection{\textbf{TODO} TODO LIST [0\%]}
\label{sec-1_2_1}

\begin{itemize}
\item $\Box$ Error/Warning managment
\item $\Box$ TODO Document every function correctly (docstrings)
\item $\Box$ TODO Check for other OS compatibility
\item $\Box$ TODO Do a validator (input file MUST be output file, and check every function)
\item $\Box$ TODO TODO tags (and others)
\item $\Box$ TODO Priority
\item $\Box$ TODO Add more types of data (List,…)
\item $\Box$ TODO Class DataStructure : Modular way to check for elements (the finding loop can became rather big if everything is implemented)
\item $\Box$ TODO Class DataStructure : Move or delete nodes
\end{itemize}
\subsubsection{BUG LIST [0\%]}
\label{sec-1_2_2}

\begin{description}
\item $\Box$ The drawers lost indentation and added spaces/tabs in properties  :NONBLOCKING::NODATALOSS:
\end{description}
\subsubsection{ChangeLog}
\label{sec-1_2_3}
\begin{itemize}

\item 0.01f
\label{sec-1_2_3_1}%
\begin{itemize}

\item New elements
\label{sec-1_2_3_1_1}%
\begin{itemize}
\item Added Schedule element for `\texttt{DEADLINE:} and `\texttt{SCHEDULED:}
\end{itemize}


\end{itemize} % ends low level
\end{itemize} % ends low level
\subsubsection{Authors [1/1]}
\label{sec-1_2_4}

\begin{itemize}
\item $\boxtimes$ BISSON Jonathan <bissonjonathan on the googlethingy>
\end{itemize}
\subsection{Code}
\label{sec-1_3}
\subsubsection{License}
\label{sec-1_3_1}
\label{31a46da7-f49b-4826-9c46-1513054f6202}
\subsubsection{Imports}
\label{sec-1_3_2}
\label{5fa2a7a6-476a-43c2-81f4-0fee4ee86fe2}
\subsubsection{Class OrgElement}
\label{sec-1_3_3}
\label{caea64f7-03b1-4f45-8abe-81819d89c6a9}
\subsubsection{Class Property}
\label{sec-1_3_4}
\label{8dec0cc1-918d-4282-8549-07efa0f3c4cc}
\subsubsection{Class Schedule}
\label{sec-1_3_5}
\label{c630bcdb-1a8c-42e0-be7d-00b291478083}
\subsubsection{Class Drawer}
\label{sec-1_3_6}
\label{72f6c28a-d103-4462-888e-297d49d0122e}
\subsubsection{Class Node}
\label{sec-1_3_7}
\label{3b4ae05e-be52-4854-a638-ecc8d2480512}
\subsubsection{Class DataStructure}
\label{sec-1_3_8}
\label{123f19bd-309b-4bda-91de-9c1ca202fac4}

\begin{verbatim}
class DataStructure:
    """
    Data structure containing all the nodes
    The root property contains a reference to the level 0 node
    """

    root = None

    def append(self,node):
        if node.parent is None: # Node has no parent (so it is the level 0 node)
            self.root = node # So parent is the first added node
        else:
            node.parent.append(node)

    def load_from_file(self,name):
        current = Node()
        parent = None
        file = open(name,'r')

        re_heading_stars = re.compile("^(\*+)\s(.*?)\s*$")
        re_drawer = re.compile("^(?:\s*?)(?::)(\S.*?)(?::)\s*(.*?)$")
        re_heading = re.compile("(?:\*+)\s((.*?)(?:\s*.*?)\s*\s)((:\S(.*?)\S:$)|$)")

        # The (?!.*?\]) protects against links containing tags being considered as tags
        re_tags = re.compile("(?:::|\s:)(\S.*?\S)(?=:)(?!.*?\])")

        re_scheduled = re.compile("(?:\s*)(SCHEDULED|DEADLINE)(?::\s*)(<.*?>)(?:\s.*|$)")

        current_drawer = None
        for line in file:
            heading_stars = re_heading_stars.search(line)
            drawer = re_drawer.search(line)
            scheduled = re_scheduled.findall(line)

            if isinstance(current, Drawer):
                if drawer:
                    if drawer.group(1) == "END":
                        current = current.parent
                    elif drawer.group(2):
                        current.append(Property(drawer.group(1),drawer.group(2)))
                else:
                    current.append(line.rstrip("\n"))
            elif drawer:
                current = current.append(Drawer(drawer.group(1)))

            elif heading_stars: # We have a heading
                self.append(current) # We append the current node as it is done

                # Is that a new level ?
                if (len(heading_stars.group(1)) > current.level): # Yes
                    parent = current # Parent is now the current node

                # If we are going back one or more levels, walk through parents
                while len(heading_stars.group(1)) < current.level:
                    current = current.parent

                # Creating a new node and assigning parameters
                current = Node() 
                current.level = len(heading_stars.group(1))
                current.heading = re_heading.findall(line)[0][0].rstrip("\n")
                current.parent = parent

                # Looking for tags
                current.tags = re_tags.findall(line)
            elif \texttt{scheduled:}

                current.append(Schedule(scheduled[0][0], scheduled[0][1]))
            else: # Nothing special, just content
                if line is not None:
                    current.append(line)

        # Add the last node
        if current.level>0:
            self.append(current)

        file.close()

    def save_to_file(self,name):
        output = open(name,'w')
        output.write(str(self.root))
        output.close()
\end{verbatim}

\end{document}